% Generated by GrindEQ Word-to-LaTeX 2012 
% ========== UNREGISTERED! ========== Please register! ==========
% LaTeX/AMS-LaTeX

\documentclass{article}

%%% remove comment delimiter ('%') and specify encoding parameter if required,
%%% see TeX documentation for additional info (cp1252-Western,cp1251-Cyrillic)
%\usepackage[cp1252]{inputenc}

%%% remove comment delimiter ('%') and select language if required
%\usepackage[english,spanish]{babel}

\usepackage{amssymb}
\usepackage{amsmath}

%%% remove comment delimiter ('%') and select graphics package
%%% for DVI output:
\usepackage[dvips]{graphicx}
%%% or for PDF output:
%\usepackage[pdftex]{graphicx}
%%% or for old LaTeX compilers:
%\usepackage[dvips]{graphics}

\begin{document}

%%% remove comment delimiter ('%') and select language if required
%\selectlanguage{spanish} 

\noindent 

\noindent 

\noindent 

\noindent 

\noindent \textbf{\textit{Head Scope}}

\noindent \textbf{\textit{Application for Head Mounted Display }}

\noindent 

\noindent \textbf{Taehun Kim}

\noindent Information System in HYU
\[2011004426\] 
chutbaksa@gmail.com

\noindent 

\noindent \textbf{Seungsun Shin}

\noindent Information System in HYU
\[2011004512\] 
vgb873@gmail.com

\noindent \textbf{Dukjin Yoon}

\noindent Information System in HYU
\[2011004534\] 
yoondukjin@outlook.com

\noindent 

\noindent \textbf{Seunggyu Jin}

\noindent Information System in HYU
\[2013012740\] 
bernardjin@naver.com

\noindent 

\textbf{\textit{Abstract}---\textit{}}

\textbf{\textit{Keywords---telepresence, Head Mounted Display}}

\textbf{\textit{Role Assignment}}

\begin{tabular}{|p{0.4in}|p{0.4in}|p{1.5in}|} \hline 
\textbf{Roles\textit{}} & \textbf{Name} & \textbf{Task description and etc.} \\ \hline 
User & Kim &  Assuming himself as a software user.\newline Discuss about software's weak point, strong point.  \\ \hline 
Customer & Shin &  Discuss about whole financial costs.  \\ \hline 
Software developer & Yoon &  Develop entire software. Consider about the program. \\ \hline 
Developer manager & Jin &  Lay out our rough sketch. Managing our plan. \\ \hline 
\end{tabular}

\textbf{\textit{}}


\section{ Introduction}

\noindent Films. Film is an incredible medium that is designed with a group of rectangle pictures that are played in a sequence. Though these pictures it can tell us stories in different ways. Film is like a window that lets you see the other world. However, our team wanted something else, we wanted something that allows you to be in the other world. It is called Virtual Reality.Virtual Reality is a machine. But through this machine it allows us to have access to another world and feel present in the world that you are inside. Our team will use the Google Cardboard, which is a box built in the shape of a snorkeling glass. Inside this Google Cardboard there are two lenses that are made of asymmetric biconvex lenses that allows one to experience a 3D reality. Along with the Google Cardboard, we will insert an android smartphone in the form of landscape and build an application that will change 2D to 3D. Inside this application we will install two functions GoogleMap and Cinema Film. Inside this Google Cardboard, it will let you feel like it is real life, and let you feel the presence of the people you are with. That is why we we chose GoogleMap and Cinema Film. With GoogleMap, it allows users to experience a full 360-degree view of the designated location. It will give them a detailed view and feel as if they are in that location. The Cinema Film will be built so that it allows the users to feel as if they were in the presence of the film. With Google Cardboard, it will let us see not a view into the world, but the whole world stretched in a rectangle. It is a form of media that can change people's perception of each other. Our team believes that Virtual Reality has the potential that can change the world. We become more empathetic, and we become more connected. Ultimately, we become more human.


\section{ Requirement}

\noindent 

\begin{enumerate}
\item  \textit{Requirement for Head Mounted Display}
\end{enumerate}

\noindent \textit{}

\begin{enumerate}
\item \textit{ }Overview 
\end{enumerate}

\noindent  A typical HMD has either one or two small displays with lenses and semi-transparent mirrors embedded in a helmet, eyeglasses or visor.

\noindent 

\begin{enumerate}
\item   Google Card Board

\item  We will use Google Cardboard for several reasons.
\end{enumerate}

\noindent 

\begin{enumerate}
\item  Google Cardboard is relatively cheaper than other HMD.
\end{enumerate}

\noindent 

\begin{enumerate}
\item  Google Cardboard could use smart phone's gyroscope function.(Because some display machines could not use gyroscope function.)
\end{enumerate}

\noindent 

\begin{enumerate}
\item  Google Cardboard has very powerful accessibility. It has very reasonable price, and many people can use it without a big burden.
\end{enumerate}

\noindent 

\begin{enumerate}
\item  Input Device
\end{enumerate}

\noindent 

\begin{enumerate}
\item  NFC Button

\item  Smart phone within gyroscope
\end{enumerate}



\begin{enumerate}
\item  Output Device

\item  Smart phone (Audio \& Screen)

\item  Show Screen through Head Mounted Display
\end{enumerate}



\begin{enumerate}
\item  Requirement for application logo

\item  We will design our application logo, using Adobe Photoshop CS6

\item  Display it on the smart phone
\end{enumerate}



\begin{enumerate}
\item  \textit{Requirement for loading page}

\item \textit{ Show our application name}

\item \textit{ Show our team name}
\end{enumerate}



\begin{enumerate}
\item  \textit{Requirement for main page}

\item \textit{ }It would  have four item\textit{}

\item \textit{ }Jumper(Telepresence to google earth API)\textit{}

\item \textit{ }Cine(Telepresence to movie viewer)\textit{}

\item \textit{ }Thinking\~{}\~{}\textit{}

\item \textit{ Requirement for Jumper}

\item \textit{ }Click the Jumper Icon among the main page\textit{}

\item \textit{ }Show the photographs of tourist attraction on the wall\textit{}

\item \textit{ }To use gyroscope, users will  find the photograph which they want to go\textit{}

\item \textit{ }Users will choose the photograph by using NFC button\textit{}

\item \textit{ }Users are able to trip the attraction zone\textit{}

\item \textit{ }If the tour is over, users could exit by back button\textit{}

\item \textit{ Requirement for Cine}

\item \textit{ }Click the Cine Icon among the main page\textit{}

\item \textit{ }Show the movie poster on the notice board\textit{}

\item \textit{ }To use gyroscope and NFC button, users will find the movie which they want to see and choose the movie\textit{}

\item \textit{ }Users could be able to watch the movie when feeling movie theater(?????? ????)\textit{}

\item \textit{ }If the movie is over, users could exit by back button\textit{}
\end{enumerate}

\noindent \textit{}

\noindent \textit{}


\section{ Development environment}

Before you begin to format your paper, first write and save the content as a separate text file. Keep your text and graphic files separate until after the text has been formatted and styled. Do not use hard tabs, and limit use of hard returns to only one return at the end of a paragraph. Do not add any kind of pagination anywhere in the paper. Do not number text heads-the template will do that for you.

Finally, complete content and organizational editing before formatting. Please take note of the following items when proofreading spelling and grammar:


\subsection{ Choice of software development platform}

\begin{enumerate}
\item \textit{ }Which platform and why? -- We will use Windows , Yosemite for android application development.
\end{enumerate}

\noindent 

\begin{enumerate}
\item  Which  program language and why? -- We will use Java(Cardboard SDK for Android) for internal functions. We also use XML for design layouts. We will use Unity for 3D Modeling.
\end{enumerate}

\noindent 

\begin{enumerate}
\item  Provide a cost estimation for your built. -- We need Photoshop CS6 \textbf{\ 323,158  }but we think it is so expensive so we use trial version of Photoshop for free. Next, we need Google Cardboard. We can buy it  \textbf{\ 5,000}. 
\end{enumerate}

\noindent 

\begin{enumerate}
\item  Provide clear information of your development environment.
\end{enumerate}

\noindent 

\noindent Android develop

\noindent -Windows7 ultimate

\noindent -OS X Yosemite 10.10.2

\noindent 

\noindent Android Studio

\noindent 

\noindent Adobe Photoshop CS6 trial version

\noindent 

\noindent Unity 5 engine


\subsection{ Software in use}

\begin{enumerate}
\item  There is Google Cardboard application already.

\item  But we 
\end{enumerate}


\subsection{ Task distribution}


\section{ Specification}

\begin{enumerate}
\item  Loading page
\end{enumerate}

\noindent 

\noindent 

\noindent 

\noindent 

\noindent 

\noindent 

\noindent 

\noindent 

\noindent 

\noindent 

\noindent 

\noindent 

\noindent 

\noindent 

\noindent 

\noindent 

\noindent 

\noindent 

\noindent 

\noindent 

\noindent 

\noindent 

\noindent 

\noindent 

\noindent 

\noindent 

\noindent 

\noindent 

\noindent 

\noindent 

\noindent 

\noindent 

\noindent 

\noindent 

\noindent 

\noindent 

\noindent 

\noindent 

\noindent 

\noindent 

\noindent 

\noindent 

\noindent 

\noindent 

\noindent 

\noindent 

\noindent 

\noindent 

\noindent 

\noindent 

\noindent 

\noindent 

\noindent 

\noindent 

\noindent 

\noindent 

\noindent 

\noindent 

\noindent 

\noindent 

\noindent 

\noindent 

\noindent 

\noindent 

\noindent 

\noindent 

\noindent 

\noindent 

\noindent 

\noindent 

\noindent 

\noindent 

\noindent 

\noindent 

\noindent 

\noindent 

\noindent 

\noindent 

\noindent 

\noindent 

\noindent 

\noindent 

\noindent 

\noindent 

\noindent 

\noindent 

\noindent 

\noindent 

\noindent 

\noindent 

\noindent 

\noindent 

\noindent 

\noindent 

\noindent 

\noindent 

\noindent 

\noindent 

\noindent 

\noindent 

\noindent 

\noindent 

\noindent 


\end{document}

% == UNREGISTERED! == GrindEQ Word-to-LaTeX 2012 ==

